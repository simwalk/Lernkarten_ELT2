\kommentar{Magnetostatik} %Thema

\begin{lk}{Wie gross ist die Magnetische Permeabilität des Vakums?}
	\LARGE{$ \mu_0 = 4 \cdot \pi \cdot 10^{-7} \dfrac{Vs}{Am} $ \\[12pt] $ = 1.2566  \cdot 10^{-6} \dfrac{H}{m}$}
\end{lk}

\begin{lk}{Was bezeichnet man als Induktivität?}
	Verhältnis zwischen zwei Grössen (Strom und Magnetischer Fluss), Proportionalitätsfaktor\\[12pt]
	\center{\huge{$ L = \dfrac{\phi}{I} $}}
\end{lk}

\begin{lk}{Welche Arten/Ausprägungen der Induktivität gibt es?}
	\begin{compactitem}
		\item Selbst Induktivität (Magnetischer Fluss welcher durch die Kontur der Fläche geht welche von einem stromdurchflossenen Leiter begrenzt wird)\\
		$ L = \dfrac{\phi_1}{I_1} $
		\item Gegen Induktivität (Ein Magnetischerfluss beeinflusst eine andere Kontur)\\
		$ L = \dfrac{\phi_2}{I_1} $
	\end{compactitem}
\end{lk}

\begin{lk}{Was ist der Unterschied zwischen innerer und äusserer Induktivität?}
	Innere Induktivität ist die Induktivität innerhalb des Leiters, äussere Induktivität ist die Induktivität ausserhalb des Leiters.
\end{lk}

\begin{lk}{Was wird als verketteter Fluss bezeichnet?}
	Wenn ein Magnetischer Fluss eine Fläche mehrfach durchdringt bezeichnet man dies als verketteter Fluss. ($ N \cdot \phi $)
\end{lk}